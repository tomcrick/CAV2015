\documentclass{llncs}
%
\usepackage{makeidx}  % allows for indexgeneration
\usepackage[british]{babel}
\usepackage{url}
\usepackage[pdftex,colorlinks=true]{hyperref}
%\fussy


\title{A Model of Reproducibility for CAV}

\author{Tom Crick\inst{1} \and Samin Ishtiaq\inst{2} \and Benjamin A. Hall\inst{3}}

\institute{Department of Computing \& Information Systems\\Cardiff Metropolitan University, UK\\
\email{tcrick@cardiffmet.ac.uk}
\and
Microsoft Research Cambridge, UK\\
\email{samin.ishtiaq@microsoft.com}
\and
MRC Cancer Unit, University of Cambridge, UK\\
\email{bh418@mrc-cu.cam.ac.uk}
}

\raggedbottom
\begin{document}
%
\frontmatter          % for the preliminaries
%
\pagestyle{headings}  % switches on printing of running heads
%\addtocmark{} % additional mark in the TOC

\maketitle

\begin{abstract}
Reliably reproducing published scientific discoveries has been acknowledged as a barrier to 
scientific progress. Even when pubhlished results reflect the authors interpretation perfectly,
novel implementation of published algorithms, and their subsequent benchmarking leaves 
substantial room for error. However, whilst this phenomena has been recognised and documented for some
time, there remains only a small subset of software available to support the specific 
needs of the research community (i.e. beyond generic tools such as git). Here we propose a 
concrete platform for reproducibility, based on a prototype which supports previously published 
work by some of the authors. The aim of this platform is to automate the build, unit testing,
and benchmarking of the BioModelAnalyzer software. We propose this prototype as a future
model for promoting and embedding reproducibility into CAV.

\end{abstract}

\section{Introduction}\label{intro}
We're proposing
stuff, based upon previous work~\cite{crick-et-al_wssspe2,crick-et-al_recomp2014}.

Cite: Recomputation Manifesto~\cite{gent:2013}, 10 Simple Rules for Reproducible
Research~\cite{sandve-et-al:2013}, Stodden~\cite{stodden-et-al:2013},
changes to dissemination in other fields e.g. psychology~\cite{chambers-et-al:2014}.

{\textbf{IDEA: pick one of Samin's CAV papers (BMA?) and run it through the process
and see what happens.}}

% http://2014.splashcon.org/track/splash2014-artifacts#About
From SPLASH: {\emph{OOPSLA Artifacts: The Artifact Evaluation Committee has been formed
to assess how well paper authors prepare artifacts in support of such
future researchers. Roughly, authors of papers who wish to participate
are invited to submit an artifact that supports the conclusions of the
paper. The Artifact Evaluation Committee will read the paper and
explore the artifact to give the authors third-party feedback about
how well the artifact supports the paper and how easy it is, in the
committee’s opinion, for future researchers to use the artifact. This
submission is voluntary and will not influence the final decision
regarding the papers. Papers that go through the Artifact Evaluation
process successfully receive a seal of approval printed on the first
page of the paper in the OOPSLA proceedings.}}

Other examples: ACM TRUST~\cite{fursin+dubach:2014}; using proofs and
Coq e.g. POPL/PLDI/ICFP.

% Q: do we need to define artefact? different from model, etc, also
% domain terminology e.g. CS vs. physics.

Big question for different research communities: How to kick this off?
Year 0: you can offer it as an optional extra, or...do it for everyone
and then present the results next year to show the issues
(carrot/stick?). There is a genuine time/research cost? Yes, but this
is the key: it has to be done and it thus part of effecting cultural
change within a community...

\begin{enumerate}
\item Call for papers: clearly advertised and high profile in conf cfp
  -- this is a new thing and a change in how we're doing stuff.
\item The game is changing, but this is (currently) extra to the
  normal reviewing process: 
e.g. {\emph{This submission is voluntary and will not influence the final decision
regarding the papers.}} -- independent of the scientific merit of the
paper, the results will be verified 
\item (prize? as well as ranked ordering at end of conf, listed in
  proceedings, badge, seal of approval, etc)
\item explicit criteria for authors? but essentially: {\emph{make this
      as easy as possible for us to evaluate/execute your artefact!}}
\item when papers are submitted, they have to nominate whether they
  want their paper to go through artefact review
\item (then you need artefact reviewers, probably taken from the pool of
  reviewers, but will need specialism)
\item specify review criteria, but essentially: {\emph{can I evaluate/execute this
  artefact and get the same results that are presented in the paper?}}
\item reporting: traffic lights and ranked list
\item at the start, this is not complusory, but this will change over a period of
time -- effecting cultural change and this would then become a
necessary condition.
\item Repo/database or previous artefacts, which would build over the
  years to give exemplars and comparisons.
\end{enumerate}

How to effectively ``generalise'' for reuse and then cascading to other communities?

\section{Conclusions}\label{concl}
We've done this for CAV...



\bibliographystyle{splncs}
\bibliography{cav2015}

\end{document}
